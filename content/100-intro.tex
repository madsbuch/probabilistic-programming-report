% Motivation
One of the tasks using probabilistic reasoning is machine learning.
Machine learning is a very broad field collecting research from many
different areas. This foundation makes it a rather daunting area to study within
as many ideas, and approaches are in play.

% Introduction
The general goal, however, is to come up with a function resembling some real world
process generating some data. The literature proposes several ways for deriving these
functions though following are, what I find to be, two central ways to achieve this.

\begin{itemize}
    \item \textbf{Inference:} The central idea here is to have a proposal
      distribution. For each observation, we condition the distribution.
      The task is to infer a posterior distribution from the prior
      distribution and all its conditioning. In \cite{Scibior:2015}, the
      authors used this approach.
    \item \textbf{Optimization:} Again, we have a proposal distribution.
      This distribution contains potentially several variables in the
      range $]0;1]$ each representing a random choice. The task is then to
      optimize these variables to yield a distribution that matches the
      proposal data.
\end{itemize}

The Hakaru project is one of the promising, relevant, techniques \cite{NarayananCRSZ16}.
They model the problem as a joint
distribution and then turn it into a conditional distribution. This approach is general,
and model developers can use to develop generative models, which is also those we focus on in this
report.

Traditionally the field has a great focus on speed over expressibility.
This goal leads to an algorithmic centric view which has no coherent
foundation. Furthermore, most algorithms are developed using discriminative
models as these are faster but more restricted \cite{Ng:2002}.
To mitigate that, a research industry of applying probabilistic
reasoning to learning tasks is emerging.

In this chapter we will look at a simple example building a distribution
from a list of elements. It is simple, yet illustrative. The central point
is that it is very hard to talk about machine learning in general terms.
The material is in plenty. Hence a project is necessary
to guide through the ocean of knowledge.


