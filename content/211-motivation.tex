\subsection{Motivation and Entropy}
In Smiths paper, \cite{Smith:2009}, he presents two expressions.
They both leak information from a variable \emph{h} (denoting
high security and should be secure) to another variable \emph{l} (denoting
low security and is publicly available).

\begin{figure}[H]
\begin{minted}{text}
if h mod 8 == 0
    then l = h
    else l = 1
\end{minted}
\caption{Example of leakage from \cite{Smith:2009}.}
\label{code:leak-1}
\end{figure}

\begin{figure}[H]
\begin{minted}{text}
l = h & 0^(7k−1)1^(k+1)
\end{minted}
\caption{Example of leakage from \cite{Smith:2009}.}
\label{code:leak-2}
\end{figure}

The first expression, Figure \ref{code:leak-1}, leaks all
data with a probability of $\frac{1}{8}$ where the other
expression, Figure \ref{code:leak-2} always leaks a fraction
of the high variable.

Using the Shannon entropy, the two programs leak equally much
information indifferent of $k$. The leak is worse \ref{code:leak-1} which motivates to
find another notion of entropy for this setting.

Smith proposes the notion of min-entropy. This notion makes sense as it
is a more conservative measure of entropy.