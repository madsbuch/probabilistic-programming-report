\section{Quantitative Information Flow}
In quantitative information flow, we are interested in how much data
flows from one entity to another. These entities can be variables in
a programming language, processes in a computer, etc.

In the setting of security, we are particularly interested in how much information
flows from secret data source to public data source of a system. These variables
can represent information in both a direct or indirect manner. 
An example of an indirect data source is the response time from another process
to derive information about some private data in the service.

In this section, we will first take a look at some clean examples to establish
the vocabulary. After that, we will look into a password checker, where we try
to derive how much information we get about the users in the database simply
by looking at timing.

