% What approach should we take?
% 1. A motivation
% 2. An introduction
% 3. An overview

% Motivation
Probabilistic reasoning is ubiquitous in computer science. It is in
all places where we do not have complete information. Here we need to
approximate, to guess or to predict some value. The applications for
probabilistic reasoning are wide, but even so, we try to delve into
some general notions of probabilistic reasoning.

For machine learning, which is the first application we delve into,
the key goal is to predict. The idea is that, given
some data, we may be able to predict the outcome of a future event. 
It should be possible to do that. After all,
it is what we constantly do as human beings.

Next, after machine learning, we will look at the applications of
probabilistic reasoning in security. Concretely we look at two areas
that rely on probabilistic techniques: Quantitative information flow
and cryptography.

The motivation for quantitative information is to figure out how much
information flows between entities in a system. These entities can
be timing modeled formally.

In Cryptography we seek to support the strength of our cryptographic
protocols. We develop these proofs through games where an adversary has
to guess better than picking at random. Here probabilistic reasoning
is at the core when defining bounds on security.

% Introduction
This technical report we will provide an overview of the current
research directions within probabilistic programming. The first section
will provide an overview of the area with a strong offset in a monadic
formulation of probabilities.

After that, we will look into select areas of probabilistic programming such
as sampling, entropy, and propositional reasoning using probabilities.

The two last parts of the report is a dive into machine learning
and how we use probabilistic programming in that setting.
This part is followed up by one on security elaborating on quantitative
information flow and cryptography.