\section{Propositional Reasoning}
% Introduction af section
Probability is a general term that many contexts use. Inferring
from samples yield a learning perspective while calculating the entropy of a
probability distribution is interesting in an information-theoretic setting. In this section,
we look into frameworks for propositional reasoning. In particular, we look
into the methodology developed for reasoning about cryptographic proofs
from the FCF paper\cite{Petcher:2015}.

% Mål for propositoinel tilgang
The key thing we reason about is the equivalence of distributions. Being able to
do this allows us to drag a pair of probability distributions into the world
of propositions. With such a notion, we can further develop the theory to
incorporate logical connectors.

% monadisk tilgang (ret, bind, rnd, repeat)
The core monadic approach used is the same as the formulation in Ramsey's
paper. Concretely they use the formulation based on measure terms here, i.e.
the continuation monad that returns a probability for a given input. In this
context, they formulate it in Gallina (Coq), as opposed to Haskell, which also 
makes sense, as we get many things out of the box.

% Diskrete distribution, endelig support, 
Finally, as this is propositional reasoning, we need to have terminating
computations to have a sound system. For this, we require discrete probability
distributions with a finite support. This requirement also yields a value of 1 when summing
the expectations of the whole support.

\subsection{Relating Distributions}
As mentioned we want a notion of equivalence for the distributions we consider.
For this, two cases have been derived as theorems, one where the distributions
are identical and one where they have a mutual bound.

% Direkte ækvivalente
For directly equal probability distributions we simply treat the distributions as pure
functions. In this context, it yields their probability mass functions. For this
property to be satisfied, we need that, for each input the PMFs yield
the same mass. Notice that this is not a bound, and as such,
the masses have to be identical.

This property is the Theorem 4 from the FCF paper.

% Inækvivalens
Furthermore, we want to reason about bounds of probabilities. To do that in this
framework they have introduced the notion of bad events that might occur in a
probability distribution. This approach is in the spirit of the identical until bad games
from \cite{Bellare:2006}.


\subsection{Program Logic}
To express richer logical statements about a distribution they develop a
Hoare logic. They design the logic in two stages. The Probabilistic
Relational Postcondition Logic (PRPL) allows to reason about pot conditions
of a probability distribution. The idea for this is to be
able to set up a formula $\Phi$ to relate two computations under:

\begin{align}
    p \sim q \{ \Phi\} 
\end{align}

Given PRPL it is straightforward to define the Probabilistic Relational
Hoare Logic (PRHL). By that, we can reason relationally about probabilities in
the following form:

\begin{align}
    \{ \Psi \} p \sim q \{ \Phi\} 
\end{align}

This section forms the foundation to build cryptographic game proofs.
