\begin{frontmatter}

%% Title, authors and addresses

%% use the tnoteref command within \title for footnotes;
%% use the tnotetext command for the associated footnote;
%% use the fnref command within \author or \address for footnotes;
%% use the fntext command for the associated footnote;
%% use the corref command within \author for corresponding author footnotes;
%% use the cortext command for the associated footnote;
%% use the ead command for the email address,
%% and the form \ead[url] for the home page:
%%
%% \title{Title\tnoteref{label1}}
%% \tnotetext[label1]{}
%% \author{Name\corref{cor1}\fnref{label2}}
%% \ead{email address}
%% \ead[url]{home page}
%% \fntext[label2]{}
%% \cortext[cor1]{}
%% \address{Address\fnref{label3}}
%% \fntext[label3]{}

\title{Probabilistic Programming}

%% use optional labels to link authors explicitly to addresses:
%% \author[label1,label2]{<author name>}
%% \address[label1]{<address>}
%% \address[label2]{<address>}

\author{Mads Buch}

\address{Aarhus, Denmark}

\begin{abstract}
Probabilistic programming is pervasive in all computations considering
incomplete data, and this is a large domain. As such,
strong abstractions and formalizations are necessary to program efficiently.
In this article, we will look into fundamental techniques for probabilistic
programming. In particular, we will take offset in the monadic structure of
probabilities.

We will apply these developed techniques in two areas: Machine Learning and
Security. In machine learning the goal task is given some amounts of data, to
make a function that fits that data. We will take offset in a simple application
considering dice rolls.

As for security, we will look into 
quantitative information flow and cryptography. Here we are respectively
interested in
limits on how much information is observable from some entity and to what
extent two distributions are equivalent. For the information flow perspective on
the entropy of a probability distribution while we take a look at some logics that allow
propositional reasoning about probability distributions to look into the
relation between two distributions.
\end{abstract}

\begin{keyword}
Probabilistic Programming Languages \sep Machine learning \sep quantitative
information flow \sep Cryptography \sep Programming Languages
%% keywords here, in the form: keyword \sep keyword

%% MSC codes here, in the form: \MSC code \sep code
%% or \MSC[2008] code \sep code (2000 is the default)

\end{keyword}

\end{frontmatter}

\newpage