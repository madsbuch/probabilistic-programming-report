\subsection{Password}
In this section, we build a model of a password checker. The scenario is as
follows: We check user credentials in a database. We do this by scanning
through the database linearly. We have detected that the interface
responds faster when the user exists in the database because we do not
need to scan through all entries.

In this scenario, an attacker can derive a username by asking the database
and see how fast it answers. As developers of the interface, we want to figure
out how much data the timing channel leaks.

The amount of data leaked is the conditional entropy of the private
variable, the user name, and the public variable, the timing. To investigate
how this can be used we build a \textbf{model} of a password checker
making the timing explicit.
For simplicity reasons we decided to use an increasing counter on
the list of user/password-pairs, we are traversing.

\begin{figure}
\begin{center}
\begin{minted}{haskell}
checkPword :: (Uname, Pword) -> [(Uname, Pword)] -> (Bool, Time)
checkPword e es = doCheckPword e es 0
  where
    doCheckPword _ [] t = (False, t)
    doCheckPword e (x : xs) t 
            | (fst x) == (fst e) =  if snd e == snd x 
                                    then (True, t)
                                    else (False, t)
            | otherwise          =  doCheckPword e xs (t+1)
\end{minted}
\end{center}
\caption{An implementation of a password checker that deliberately leaks time}
\label{code:check/pword}
\end{figure}

The code in Figure \ref{code:check/pword} shows the implementation of a password
checker. It enumerates a list until it finds a hit. Here it checks the
password not explicitly reporting whether the user is in the table. If no hit
is found it returns False.

The public variable in our model is the time. We have made the time explicit as
an increasing number depending on where the element is in the list.
We build a probabilistic model of the private and public variables.
For this, it is evident that the public variable is dependent on the private
one.

\begin{figure}[H]
\begin{center}
\begin{minted}{haskell}
-- Public
pTime pUname = do 
    pword <- pStringNormal -- We don't care about the pword
    uname <- pUname
    return (snd ( checkPword (uname, pword) pWords) )

-- Private
pUname = pStringNormal
\end{minted}
\end{center}
\caption{Probabilistic modeling of the private and public data}
\label{code:pword-model}
\end{figure}

In Figure \ref{code:pword-model} we have the implementation of the model.
As visible we directly use the implementation of the password checker, and
as such, we are always sure that our probabilistic model corresponds to
our actual model of a given situation.

After the modeling, we can attempt to derive the entropy. To make it feasible
we decided to reduce to the domain of the strings to two characters. Already at three
characters, the queries took too long to be feasible on our computers with a
sequential implementation.

%Min Entropy
%0.9911242603550298
%Shannon Entropy
%9.618774814425457e-2

The derived Shannon entropy was roughly \textbf{0.09618} while the min-entropy
yielded approximately \textbf{0.99112}. This difference indicates, as expected that
the min-entropy reports a more conservative measure than the Shannon entropy.

\textbf{A remark on modeling and sampling:} When implementing this example I
had to implement uniform distributions over strings. Here we have a couple of
ways to do it. The naive way would be something like following.

\begin{figure}[H]
\begin{center}
\begin{minted}{haskell}
pThreeLetterStringOld = uniform [[x1, x2, x3] | 
    x1 <- ['a' .. 'z'],
    x2 <- ['a' .. 'z'],
    x3 <- ['a' .. 'z']]
\end{minted}
\end{center}
\caption{An inefficient implementation of a uniform distribution over a 3 character string}
\label{code:uniform-string-slow}
\end{figure}

The idea is to make a list of all three-letter strings and then make it to a
uniform distribution.
This construction is not efficient for sampling as we get a list of 17576
elements we potentially have to run through.

\begin{figure}[H]
\begin{center}
\begin{minted}{haskell}
pStringNormal = do
    x1 <- uniform ['a' .. 'z']
    x2 <- uniform ['a' .. 'z']
    x3 <- uniform ['a' .. 'z']
    return [x1, x2, x3]
\end{minted}
\end{center}
\caption{An efficient implementation of a uniform distribution over a 3 character string}
\label{code:uniform-string-fast}
\end{figure}

An alternative implementation in this setting would be the one in Figure \ref{code:uniform-string-fast}.
This method is strictly faster to sample from, as we have three lists of 26 elements
summing to a maximum of 78 elements.

The speed increase does have a cost as we use three times more randomness.

Another thing we also have to note is that exact expectation querying takes
the same time in both implementations. However, if we do keep the recommendations
about sampling in mind, the second implementation is preferred.