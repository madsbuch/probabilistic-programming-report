%% This is file `elsarticle-template-1-num.tex',
%%
%% Copyright 2009 Elsevier Ltd
%%
%% This file is part of the 'Elsarticle Bundle'.
%% ---------------------------------------------
%%
%% It may be distributed under the conditions of the LaTeX Project Public
%% License, either version 1.2 of this license or (at your option) any
%% later version.  The latest version of this license is in
%%    http://www.latex-project.org/lppl.txt
%% and version 1.2 or later is part of all distributions of LaTeX
%% version 1999/12/01 or later.
%%
%% The list of all files belonging to the 'Elsarticle Bundle' is
%% given in the file `manifest.txt'.
%%
%% Template article for Elsevier's document class `elsarticle'
%% with numbered style bibliographic references
%%
%% $Id: elsarticle-template-1-num.tex 149 2009-10-08 05:01:15Z rishi $
%% $URL: http://lenova.river-valley.com/svn/elsbst/trunk/elsarticle-template-1-num.tex $
%%
\documentclass[preprint,12pt]{elsarticle}

\makeatletter
\def\ps@pprintTitle{%
   \let\@oddhead\@empty
   \let\@evenhead\@empty
   \let\@oddfoot\@empty
   \let\@evenfoot\@oddfoot
}
\makeatother

% \documentclass[preprint,12pt]{elsarticle}

%% Use the option review to obtain double line spacing
%% \documentclass[preprint,review,12pt]{elsarticle}

%% Use the options 1p,twocolumn; 3p; 3p,twocolumn; 5p; or 5p,twocolumn
%% for a journal layout:
%% \documentclass[final,1p,times]{elsarticle}
%% \documentclass[final,1p,times,twocolumn]{elsarticle}
%% \documentclass[final,3p,times]{elsarticle}
%% \documentclass[final,3p,times,twocolumn]{elsarticle}
%% \documentclass[final,5p,times]{elsarticle}
%% \documentclass[final,5p,times,twocolumn]{elsarticle}

%% if you use PostScript figures in your article
%% use the graphics package for simple commands
%% \usepackage{graphics}
%% or use the graphicx package for more complicated commands
%% \usepackage{graphicx}
%% or use the epsfig package if you prefer to use the old commands
%% \usepackage{epsfig}

\usepackage{amsmath}
%% The amssymb package provides various useful mathematical symbols
\usepackage{amssymb}
%% The amsthm package provides extended theorem environments
\usepackage{amsthm}

%% The lineno packages adds line numbers. Start line numbering with
%% \begin{linenumbers}, end it with \end{linenumbers}. Or switch it on
%% for the whole article with \linenumbers after \end{frontmatter}.
\usepackage{lineno}
\usepackage{marginnote}
\usepackage{minted}
\usepackage{hyperref}
\usepackage{epigraph}

%% natbib.sty is loaded by default. However, natbib options can be
%% provided with \biboptions{...} command. Following options are
%% valid:

%%   round  -  round parentheses are used (default)
%%   square -  square brackets are used   [option]
%%   curly  -  curly braces are used      {option}
%%   angle  -  angle brackets are used    <option>
%%   semicolon  -  multiple citations separated by semi-colon
%%   colon  - same as semicolon, an earlier confusion
%%   comma  -  separated by comma
%%   numbers-  selects numerical citations
%%   super  -  numerical citations as superscripts
%%   sort   -  sorts multiple citations according to order in ref. list
%%   sort&compress   -  like sort, but also compresses numerical citations
%%   compress - compresses without sorting
%%
%% \biboptions{comma,round}

% \biboptions{}


%\journal{Journal Name}

\begin{document}

\begin{frontmatter}

%% Title, authors and addresses

%% use the tnoteref command within \title for footnotes;
%% use the tnotetext command for the associated footnote;
%% use the fnref command within \author or \address for footnotes;
%% use the fntext command for the associated footnote;
%% use the corref command within \author for corresponding author footnotes;
%% use the cortext command for the associated footnote;
%% use the ead command for the email address,
%% and the form \ead[url] for the home page:
%%
%% \title{Title\tnoteref{label1}}
%% \tnotetext[label1]{}
%% \author{Name\corref{cor1}\fnref{label2}}
%% \ead{email address}
%% \ead[url]{home page}
%% \fntext[label2]{}
%% \cortext[cor1]{}
%% \address{Address\fnref{label3}}
%% \fntext[label3]{}

\title{Probabilistic Programming}

%% use optional labels to link authors explicitly to addresses:
%% \author[label1,label2]{<author name>}
%% \address[label1]{<address>}
%% \address[label2]{<address>}

\author{Mads Buch}

\address{Aarhus, Denmark}

\begin{abstract}
Probabilistic programming is pervasive in all computations considering
incomplete data, and this is a large domain. As such,
strong abstractions and formalizations are necessary to program efficiently.
In this article, we will look into fundamental techniques for probabilistic
programming. In particular, we will take offset in the monadic structure of
probabilities.

We will apply these developed techniques in two areas: Machine Learning and
Security. In machine learning the goal task is given some amounts of data, to
make a function that fits that data. We will take offset in a simple application
considering dice rolls.

As for security, we will look into 
quantitative information flow and cryptography. Here we are respectively
interested in
limits on how much information is observable from some entity and to what
extent two distributions are equivalent. For the information flow perspective on
the entropy of a probability distribution while we take a look at some logics that allow
propositional reasoning about probability distributions to look into the
relation between two distributions.
\end{abstract}

\begin{keyword}
Probabilistic Programming Languages \sep Machine learning \sep quantitative
information flow \sep Cryptography \sep Programming Languages
%% keywords here, in the form: keyword \sep keyword

%% MSC codes here, in the form: \MSC code \sep code
%% or \MSC[2008] code \sep code (2000 is the default)

\end{keyword}

\end{frontmatter}

\newpage

%%
%% Start line numbering here if you want
%%
%\linenumbers

%% main text
\part{Overview}

% What approach should we take?
% 1. A motivation
% 2. An introduction
% 3. An overview

% Motivation
Probabilistic reasoning is ubiquitous in computer science. It is in
all places where we do not have complete information. Here we need to
approximate, to guess or to predict some value. The applications for
probabilistic reasoning are wide, but even so, we try to delve into
some general notions of probabilistic reasoning.

For machine learning, which is the first application we delve into,
the key goal is to predict. The idea is that, given
some data, we may be able to predict the outcome of a future event. 
It should be possible to do that. After all,
it is what we constantly do as human beings.

Next, after machine learning, we will look at the applications of
probabilistic reasoning in security. Concretely we look at two areas
that rely on probabilistic techniques: Quantitative information flow
and cryptography.

The motivation for quantitative information is to figure out how much
information flows between entities in a system. These entities can
be timing modeled formally.

In Cryptography we seek to support the strength of our cryptographic
protocols. We develop these proofs through games where an adversary has
to guess better than picking at random. Here probabilistic reasoning
is at the core when defining bounds on security.

% Introduction
This technical report we will provide an overview of the current
research directions within probabilistic programming. The first section
will provide an overview of the area with a strong offset in a monadic
formulation of probabilities.

After that, we will look into select areas of probabilistic programming such
as sampling, entropy, and propositional reasoning using probabilities.

The two last parts of the report is a dive into machine learning
and how we use probabilistic programming in that setting.
This part is followed up by one on security elaborating on quantitative
information flow and cryptography.

% Basic PL view of probabilities
% A Probabilistic Language Based on Sampling Functions
\section{Monadic Structure}
Monads make out a good choice to encapsulate probability distributions.
They provide a natural way to lift values to a distribution and to compose
distributions from existing ones.

In this section, we seek to investigate probabilities using
monads. The core structure is identical for all of them, but other than that,
there is no consensus on these frameworks. We will look at
some of the ideas and try to compare them.

The ideas are from following papers \cite{Ramsey:2002, Scibior:2015, Petcher:2015, Park:2008}.
\cite{Ramsey:2002, Scibior:2015} both use a purely monadic style implemented in Haskell,
while \cite{Petcher:2015} implemented it in Coq. Both of these choices make good sense taking
their goals into consideration. 
In \cite{Park:2008} they develop a language
which structure resembles the monad.

The core functions implemented for monads are \emph{bind}
and \emph{return}. They are the same in both Ramsey's, and
Scibior's implementations, namely that \emph{bind} resembles a composition and
return is the Dirac distribution. Park implemented bind through
sampling with all expressions 
lifted through the use of the \emph{prob} construction.

The differences lay in what auxiliary functions their frameworks provide.

\subsection{Operations}
After having defined the core monadic structure, we implement
some operations that let us perform some useful tasks with monads. They are
different depending on who you ask, their application,
and what constraints they impose.

For learning tasks of large domains, we want to be able to sample because it is infeasible
to consider the full domain. In information flow, we want to be able to
derive the entropy of a distribution. For this, we need a \emph{support} query and an
\emph{expectation} query. For sufficiently large domains we will also use the Monte
Carlo methods for expectation and support, but more on that in the sampling section.

\subsection{Implementations}
Having looked into the monad approach to probability distributions, we will now
elaborate on the implementations. We will take a look at their
design motivations and how they compare.

Fundamentally all the monads have the same core functions. That is, the
\emph{return} operator is the Dirac distribution that forms a distribution of
a single value with an expectation of 1. The \emph{bind} operator amounts to
draw from one distribution to constitute another distribution.

Now, these two constructions do not introduce random choices in any ways.
Their purpose is primarily there for modeling reasons.

The following implementations are interesting in the way they tackle the task
of introducing randomness, and what features they emphasize.

% Ramseys monad - descrete but easy to express exact operations
% Scibior       - Clean and based on sampling and conditioning
% Petcher       - emphasis on

% Parks         - Emphasis on sampling, but unclean architecture

% RAMSEY

Ramsey's monad \cite{Ramsey:2002} augments the standard monad constructors by
a choice operator. The operator takes a probability, $p$. This 
probability is a number between zero and one. It then returns one of two
other expressions with probability respectively $p$ and $p-1$.

The \emph{choose} operator is simple in the sense that it is how we think
about stochastic lambda calculus. It makes it possible to implement the
\emph{support} query, and the \emph{expectation} query using exact implementations.
This is desirable for any application using the entropy of a distribution or
otherwise in need of these queries. They are, however, slow for
large distributions which should not come as a surprise, as the \emph{choose}
operator introduces probability distributions with exponential fan-out.

Furthermore, it is possible to implement the \emph{sample} query widely
used in any applications that need approximate queries.

\begin{figure}[H]
\begin{minted}{haskell}
data P a where
    Return :: a -> P a
    Bind   :: P a -> (a -> P b) -> P b -- The reason for GADTs
    Choose :: Probability -> P a -> P a -> P a
\end{minted}
\caption{The algebraic definition of the probability monad from \cite{Ramsey:2002}.}
\label{code:ramsey-monad}
\end{figure}

Figure \ref{code:ramsey-monad} shows the implementation using monadic terms. We
could also have implemented it using measures. This way would, however, 
pose a limit on implementing the \emph{support} query. Such an implementation
would include enumerating the entire domain of the probability distribution. for
each element we would need to calculate the expectation and exclude it if the
expectation is 0.

Lastly, we mention that this implementation only works for discrete
distributions, again, due to the \emph{choose} operator. That choice gives
rise to some applications, e.g., propositional reasoning, but makes other applications
harder.

% SCIBIOR
Next, we look at Scibiors implementation \cite{Scibior:2015}.

This implementation extends the monad with a \emph{primitive} operator and
a \emph{conditional} operator. This implementation is from the paper titled
\emph{Practical Probabilistic Programming with Monads} and has a heavy focus
on the machine learning aspect of probabilistic programming. As such, none of these
added operators provide a good fit for exact queries. They do, however, make it
much easier to do sampling and conditioning. It is even possible to encode
continuous distributions in this implementation.

\begin{figure}[H]
\begin{minted}{haskell}
data Dist a where
    Return      :: a -> Dist a
    Bind        :: Dist b -> (b -> Dist a) -> Dist a
    Primitive   :: Sampleable d => d a -> Dist a
    Conditional :: (a -> Prob) -> Dist a -> Dist a
\end{minted}
\caption{The algebraic definition of the probability monad from \cite{Scibior:2015}.}
\label{code:scibior-monad}
\end{figure}

The code in Figure \ref{code:scibior-monad} is the algebraic definition of the code
from the paper. The two first cases are not particularly interesting.
The \emph{Primitive} constructor takes anything that implements the \emph{Sampleable}
type class and lifts it into the monad. This lifting includes external functions
and is the reason why we can make arbitrary distributions.

The \emph{Conditional} constructor is used to condition a probability distribution.
To do this,
we need a function calculating the likelihood of an element and a prior
probability distribution.
We then get a conditioned distribution. It is not possible to sample from that.
First, we need to \emph{infer} the posterior distribution.

Next, we have an implementation with a goal of supporting propositional reasoning.
Aside from the usual constructors, we consider the \emph{Rnd} and the \emph{Repeat}
constructors. The biggest difference from this to the others is, that
it makes randomness completely explicit when we ask for a vector of random bits.

\begin{figure}[H]
\begin{minted}{coq}
Inductive Comp : Set -> Type :=
| Ret : forall {A : Set}{H: EqDec A}, A -> Comp A
| Bind : forall {A B : Set}, Comp B -> (B -> Comp A) -> Comp A
| Rnd : forall n, Comp (Bvector n)
| Repeat : forall {A : Set}, Comp A -> (A -> bool) -> Comp A.
\end{minted}
\caption{The algebraic definition of the probability monad in \cite{Petcher:2015}.}
\label{code:petcher-monad}
\end{figure}

Figure \ref{code:petcher-monad} shows the implementation used in their paper.
The \emph{Ret} and \emph{Bind} are the usual monad constructors for the probability
monad.
\emph{Rnd} is the constructor
that exposes randomness. It returns an \emph{n}-sized vector of random bits, which then
has to be utilized by the application routines. The last constructor, \emph{Repeat}, 
is used to do a kind of conditioning on a distribution. This conditioning could aid
sampling in some interval on a distribution over natural numbers.

% PARK
The last implementation is from \cite{Park:2008}. They went with a complete formulation
of a language rather than implementing it as a DSL in an existing language.
They base the language on the idea of lifting entire terms into a \emph{circle}-type
on which we can do sampling.

Other operations they implemented in the language are the \emph{expectation} query
and the \emph{Bayes} operation, which is essentially inference. They are both
implemented through sampling and need an argument on how much data to
sample to approximate the queries.
We may embed both of the queries in a probabilistic term, e.g. we can do an
expectation query within a distribution.

Furthermore, the sampling based algorithms use a global randomness pool.
This global approach results in computations with side effects which
are notoriously hard to reason about.

While this might seem like a good idea, it makes it hard to reason about the
distribution as a whole and certainly breaks compositionality. Therefore
this approach is mostly here for completeness and reference.

% Discussion and comparison

\textbf{Final remarks:} All above implementations are different. From here it is hard
to say whether one is better than another as they all suit different goals. The one
by Ramsey emphasizes a general purpose implementation that is conceptually simple.
However, they did not emphasize inference.

Scibior made a monad with the particular aim of being specialized to inference
based on sampling. They indeed succeeded in that, but this implementation
would not be useful in the security context where rigorous introspection or
entropy of large distributions is the question. Hence he made a good
implementation for Machine learning.

Lastly, Petcher made a monad specialized towards doing formal reasoning.
Among other things, the way he handles randomness is very suited this goal.
The monad provides randomness as a vector of discrete values. This approach
makes it well suited for the purpose. However, it is not natural to express
random computations. % Stochastic Lambda Calculus and Monads of Probability Distributions
\section{Semantics}
After the above formulations of probability distributions, a natural question
is to what extent we can express the probability distributions we
usually do. To answer this question, we look into the semantics of the
probability distributions

In \cite{Ramsey:2002}, they develop an interpretation of the probability through
stochastic lambda calculus. This translation shows the generality of the probability
monad. In his monad, we can implement the usual stochastic lambda
calculus. This interpretation should not come as a surprise as he
augments his monad with the
\emph{choose} operator which tightly resembles a stochastic choice.

Both Ramsey and Scibior interpret the monad into measure terms.
This interpretation yields an argument of the soundness of their monadic approaches.

We will not look further into the formal definitions other than referring
to \cite{Ramsey:2002, Scibior:2015}.

% Select operations
%\section{Disintegration}
%The motivation for disintegration stems from the problem of zero-measures
%in a dimension of a probability distribution. Expectations of probability
%distributions on continuous spaces is defined to be the volume of the
%object the distribution spans. This, however, poses a problem, when we have
%observations on a given point: Points do not have any volume in a continuous
%space.

%This problem can be solved using disintegration \cite{Shan:2016}.
%
%\subsection{Formal Semantics}
%A formal semantics of disintegration interprets the transformation into
 % Symbolic Bayesian Inference by Lazy Partial Evaluation
\section{Sampling}
In this section, we look a bit further into sampling. Sampling is necessary
because it is the only method that seems tractable for arbitrary
distributions. Hence we wish to be able to express the other queries through
sampling. This idea is fundamental for \cite{Park:2008, Mansinghka:2009, Scibior:2015}

The issue on sampling also tabs into the difference between
\emph{discriminative} and \emph{generative} models.
The approach described in this text takes offset in generative models. A complete comparison of the two
approaches is out of scope for the report. However, the go-to paper on this
is \cite{Ng:2002}. The conclusion of the article is that for performance oriented applications
we should use discriminative models. Discriminative models solely
model conditional distributions and therefore are not general enough to capture
all uses of probabilities.

The motivation is the tractability of doing exact calculations of certain properties
about probability distributions. Given a normally distributed five character string (in the
interval 'aaaaa' .. 'zzzzz'), I can sample at a rate of 22MB (~3MSamples) per second
while it takes 3.348 seconds to calculate the exact expectation of an element in
the distribution.

The Main two points we emphasize is that it takes linear time to sample (in a graphical
model) where it takes exponential time to do exact queries (traversing trees). This complexity difference,
naturally, comes with a trade-off; We can only do approximate queries from sampling.

For machine learning applications sampling is usually fine. We need approximate
queries. They just have to answer well within range. On the other hand, doing
formal reasoning makes it strictly harder to use sampling. The conclusion is that
sampling is useful in empiric domains but it is not ideal in 
strictly formal domains. % Natively Probabilistic Computation
\section{Entropy}
Sometimes we are interested in how random a distribution is. We
cover this question under the term entropy. The notion entropy often defaults
to what was defined by Shannon, i.e. what is the lower bound of
the number of bits we need to do a transmission. However, other notions exist,
defined on some other metrics being more useful in some applications.

For this report, we will look at two types of entropy: Shannon entropy and
min-entropy. They differ in how they fundamentally approach the question
of entropy in information science and their applications.

The machine learning community widely uses the Shannon entropy
where we seek to decrease the entropy as much as possible to a concise
description of some data. Its origins as a mean to discuss compression
of data, and as such it is used in compression based models.

On the other hand, min-entropy is proposed to be the notion used in
information flow as it takes a more conservative approach to the
concept of entropy.

Following to this is a presentation on the implementations showing
how the entropy discussion fits into the Ramsey's probability monad.
It will be evident that it is indeed possible to implement it and
use it in real applications for sufficiently small distributions.

\subsection{Implementations}
We used Ramsey's probability monad to calculate entropies.
This was because their monad supports the \emph{support}
and the \emph{expectation} query. These can also be implemented in
an exact manner in Petchers monad but will be hard to realize in
Scibiors. In Scibiors monad we would approximate entropy
through sampling. This approximation becomes infeasible
when the domain of the distribution gets bigger.

% Implementation of shannon entropy

\begin{figure}[H]
\begin{minted}{haskell}
shannon :: (Eq a, Ord a) => P a -> Double
shannon px = 
  let
      exp x = expectThat px x
   in
      sum ( map (\x -> (exp x) * 
                       (logBase 2 $ (1 / exp x))) $ supp px )
\end{minted}
\caption{The implementation of the Shannon entropy as written in \cite{Smith:2009}.}
\label{code:shannon}
\end{figure}

The code in Figure \ref{code:shannon} shows the Haskell implementation of the
Shannon entropy. It uses some functions not explicitly defined though it should be
immediate from their names what they do. From the last line, it is apparent that it
maps over the full domain of the distribution and sums the result. Though this
problem is embarrassingly parallel, its complexity still outperforms computing power
for interesting distributions.

% Implementation of min entropy
Above implementation was proposed bad for the application of quantitative information
flow as it classified. Instead, Smith proposes to use the min-entropy notion for evaluating
distributions and their information leakage \cite{Smith:2009}.

\begin{figure}[H]
\begin{minted}{haskell}
condVuln :: (Ord x, Ord y) => (P y -> P x) -> P y -> Double
condVuln px py = sum [(expectThat py y) * 
                      (vuln $ px $ return y) | y <- supp py]

condMinEntropy px py = logBase 2 (1/(condVuln pl ph))
\end{minted}
\caption{The implementation of the the entropy entropy as written in \cite{Smith:2009}.}
\label{code:min-entropy}
\end{figure}

The implementation in Figure \ref{code:min-entropy} shows how we implemented min-entropy through
Ramsey's probability monad. Again it completely resembles the implementation
from \cite{Smith:2009} and it has the same complexity concerns as the other implementation.



% Constructive and propositional way of thinking of probabilities.
%\section{Inference}
%
%The term \emph{inference} in the probability theoretic vocabulary covers the
%process of moving a prior distribution to a posterior distribution based on
%empirics. These empirics are usually in the form of observed data. 
\section{Propositional Reasoning}
% Introduction af section
Probability is a general term that many contexts use. Inferring
from samples yield a learning perspective while calculating the entropy of a
probability distribution is interesting in an information-theoretic setting. In this section,
we look into frameworks for propositional reasoning. In particular, we look
into the methodology developed for reasoning about cryptographic proofs
from the FCF paper\cite{Petcher:2015}.

% Mål for propositoinel tilgang
The key thing we reason about is the equivalence of distributions. Being able to
do this allows us to drag a pair of probability distributions into the world
of propositions. With such a notion, we can further develop the theory to
incorporate logical connectors.

% monadisk tilgang (ret, bind, rnd, repeat)
The core monadic approach used is the same as the formulation in Ramsey's
paper. Concretely they use the formulation based on measure terms here, i.e.
the continuation monad that returns a probability for a given input. In this
context, they formulate it in Gallina (Coq), as opposed to Haskell, which also 
makes sense, as we get many things out of the box.

% Diskrete distribution, endelig support, 
Finally, as this is propositional reasoning, we need to have terminating
computations to have a sound system. For this, we require discrete probability
distributions with a finite support. This requirement also yields a value of 1 when summing
the expectations of the whole support.

\subsection{Relating Distributions}
As mentioned we want a notion of equivalence for the distributions we consider.
For this, two cases have been derived as theorems, one where the distributions
are identical and one where they have a mutual bound.

% Direkte ækvivalente
For directly equal probability distributions we simply treat the distributions as pure
functions. In this context, it yields their probability mass functions. For this
property to be satisfied, we need that, for each input the PMFs yield
the same mass. Notice that this is not a bound, and as such,
the masses have to be identical.

This property is the Theorem 4 from the FCF paper.

% Inækvivalens
Furthermore, we want to reason about bounds of probabilities. To do that in this
framework they have introduced the notion of bad events that might occur in a
probability distribution. This approach is in the spirit of the identical until bad games
from \cite{Bellare:2006}.


\subsection{Program Logic}
To express richer logical statements about a distribution they develop a
Hoare logic. They design the logic in two stages. The Probabilistic
Relational Postcondition Logic (PRPL) allows to reason about pot conditions
of a probability distribution. The idea for this is to be
able to set up a formula $\Phi$ to relate two computations under:

\begin{align}
    p \sim q \{ \Phi\} 
\end{align}

Given PRPL it is straightforward to define the Probabilistic Relational
Hoare Logic (PRHL). By that, we can reason relationally about probabilities in
the following form:

\begin{align}
    \{ \Psi \} p \sim q \{ \Phi\} 
\end{align}

This section forms the foundation to build cryptographic game proofs.


\section{Research Questions}
This area is still emerging, and as such, much of the material is rather new.
Following is a select list of research questions from the literature. It
should provide an idea of the current state of this field.

\begin{itemize}
    \item \textbf{Operations based on sampling:} Evidently it is not tractable
      to implement exact methods of certain operations. Hence we need to
      approximate them. A neat foundation for this would be sampling as it is
      cheap and does not require any introspection on the distribution.
    \item \textbf{Parallelization:} This is an extension of above. The process
      should be fast enough to achieve good results in feasible time.
      Parallelization is a hot topic, and plenty of specialized hardware for the
      purpose exist. Hence efficient parallel implementation of sampling based
      operations would push this field into a useful state.
    \item \textbf{Formal semantics:} The body of work is quite small in this area of
      providing a formal semantics for these constructions. A formal semantics
      would increase the reliability on the constructions and provide deeper
      insights into their applications.
\end{itemize}

Many more questions exist. A good starting point is \cite{Scibior:2015}.

\newpage
\part{Machine Learning}
\epigraph{Don't calculate probabilities; 
  Sample good guesses.}{\textit{Vikash Kumar \cite{Mansinghka:2009}}}
% Applications of probabilistic reasoning where inference is its
% core
% Practical Probabilistic Programming with Monads
% Motivation
One of the tasks using probabilistic reasoning is machine learning.
Machine learning is a very broad field collecting research from many
different areas. This foundation makes it a rather daunting area to study within
as many ideas, and approaches are in play.

% Introduction
The general goal, however, is to come up with a function resembling some real world
process generating some data. The literature proposes several ways for deriving these
functions though following are, what I find to be, two central ways to achieve this.

\begin{itemize}
    \item \textbf{Inference:} The central idea here is to have a proposal
      distribution. For each observation, we condition the distribution.
      The task is to infer a posterior distribution from the prior
      distribution and all its conditioning. In \cite{Scibior:2015}, the
      authors used this approach.
    \item \textbf{Optimization:} Again, we have a proposal distribution.
      This distribution contains potentially several variables in the
      range $]0;1]$ each representing a random choice. The task is then to
      optimize these variables to yield a distribution that matches the
      proposal data.
\end{itemize}

The Hakaru project is one of the promising, relevant, techniques \cite{NarayananCRSZ16}.
They model the problem as a joint
distribution and then turn it into a conditional distribution. This approach is general,
and model developers can use to develop generative models, which is also those we focus on in this
report.

Traditionally the field has a great focus on speed over expressibility.
This goal leads to an algorithmic centric view which has no coherent
foundation. Furthermore, most algorithms are developed using discriminative
models as these are faster but more restricted \cite{Ng:2002}.
To mitigate that, a research industry of applying probabilistic
reasoning to learning tasks is emerging.

In this chapter we will look at a simple example building a distribution
from a list of elements. It is simple, yet illustrative. The central point
is that it is very hard to talk about machine learning in general terms.
The material is in plenty. Hence a project is necessary
to guide through the ocean of knowledge.



\section{Dice}
To have an illustrative example of a learning task, we will attempt to learn the
distribution of a dice based on a list of dice rolls. We use frequencies as our
main approach. This method is useful when considering large amounts of data.

\begin{figure}
\begin{minted}{haskell}
observe :: a -> (Int, P a) -> (Int, P a)
observe e (t, prior) = let t' = t+1 in
        (t'+1, choose (1/(fromIntegral t')) (return e) prior)
\end{minted}
\caption{A function that takes an observation and adjust the distribution.}
\label{code:dice}
\end{figure}

In \ref{code:dice} the function \emph{observe} takes an element and a distribution
of the same type as the element. The input distribution is augmented by a count
of elements already observed. Now we choose that element by $\frac{1}{N}$.
Doing this yields a uniformly distributed distribution over all elements it
has observed, given that all elements are unique. If multiple elements are the
same, we sum their expectations.

We wrapped the probability distributions in a tuple of the distribution and the
total count of elements it has observed. We need this total count to calculate
how much impact the new element should have. In this approach we let the impact
of the new element be inversely proportional to the number of elements we already
saw. As we also noted, this yields an inductive uniform distribution.

\newpage
\part{Security}
% The area where propositional reasoning is at core
%Introduction
Probabilistic reasoning is a fundamental technique for assessing security risks.
In this part, we will look into two security relevant subjects: Quantitative
information flow and cryptography.

In quantitative information flow, we are interested in how much
information flows from one entity to another in a system. To uncover
covert channels we consider models of the computations.

In cryptography, we are interested in assessing the difficulty in retrieving
the clear text of some encrypted string. Here we are interested in doing
proofs that reduce the hardness of a protocol to something that we
consider hard. For this, we use the notion of game-based security proofs.
These can be formalized using the monadic approach presented in Part 1.

% On the Foundations of Quantitative Information Flow
\section{Quantitative Information Flow}
In quantitative information flow, we are interested in how much data
flows from one entity to another. These entities can be variables in
a programming language, processes in a computer, etc.

In the setting of security, we are particularly interested in how much information
flows from secret data source to public data source of a system. These variables
can represent information in both a direct or indirect manner. 
An example of an indirect data source is the response time from another process
to derive information about some private data in the service.

In this section, we will first take a look at some clean examples to establish
the vocabulary. After that, we will look into a password checker, where we try
to derive how much information we get about the users in the database simply
by looking at timing.


\subsection{Motivation and Entropy}
In Smiths paper, \cite{Smith:2009}, he presents two expressions.
They both leak information from a variable \emph{h} (denoting
high security and should be secure) to another variable \emph{l} (denoting
low security and is publicly available).

\begin{figure}[H]
\begin{minted}{text}
if h mod 8 == 0
    then l = h
    else l = 1
\end{minted}
\caption{Example of leakage from \cite{Smith:2009}.}
\label{code:leak-1}
\end{figure}

\begin{figure}[H]
\begin{minted}{text}
l = h & 0^(7k−1)1^(k+1)
\end{minted}
\caption{Example of leakage from \cite{Smith:2009}.}
\label{code:leak-2}
\end{figure}

The first expression, Figure \ref{code:leak-1}, leaks all
data with a probability of $\frac{1}{8}$ where the other
expression, Figure \ref{code:leak-2} always leaks a fraction
of the high variable.

Using the Shannon entropy, the two programs leak equally much
information indifferent of $k$. The leak is worse \ref{code:leak-1} which motivates to
find another notion of entropy for this setting.

Smith proposes the notion of min-entropy. This notion makes sense as it
is a more conservative measure of entropy.
\subsection{Password}
In this section, we build a model of a password checker. The scenario is as
follows: We check user credentials in a database. We do this by scanning
through the database linearly. We have detected that the interface
responds faster when the user exists in the database because we do not
need to scan through all entries.

In this scenario, an attacker can derive a username by asking the database
and see how fast it answers. As developers of the interface, we want to figure
out how much data the timing channel leaks.

The amount of data leaked is the conditional entropy of the private
variable, the user name, and the public variable, the timing. To investigate
how this can be used we build a \textbf{model} of a password checker
making the timing explicit.
For simplicity reasons we decided to use an increasing counter on
the list of user/password-pairs, we are traversing.

\begin{figure}
\begin{center}
\begin{minted}{haskell}
checkPword :: (Uname, Pword) -> [(Uname, Pword)] -> (Bool, Time)
checkPword e es = doCheckPword e es 0
  where
    doCheckPword _ [] t = (False, t)
    doCheckPword e (x : xs) t 
            | (fst x) == (fst e) =  if snd e == snd x 
                                    then (True, t)
                                    else (False, t)
            | otherwise          =  doCheckPword e xs (t+1)
\end{minted}
\end{center}
\caption{An implementation of a password checker that deliberately leaks time}
\label{code:check/pword}
\end{figure}

The code in Figure \ref{code:check/pword} shows the implementation of a password
checker. It enumerates a list until it finds a hit. Here it checks the
password not explicitly reporting whether the user is in the table. If no hit
is found it returns False.

The public variable in our model is the time. We have made the time explicit as
an increasing number depending on where the element is in the list.
We build a probabilistic model of the private and public variables.
For this, it is evident that the public variable is dependent on the private
one.

\begin{figure}[H]
\begin{center}
\begin{minted}{haskell}
-- Public
pTime pUname = do 
    pword <- pStringNormal -- We don't care about the pword
    uname <- pUname
    return (snd ( checkPword (uname, pword) pWords) )

-- Private
pUname = pStringNormal
\end{minted}
\end{center}
\caption{Probabilistic modeling of the private and public data}
\label{code:pword-model}
\end{figure}

In Figure \ref{code:pword-model} we have the implementation of the model.
As visible we directly use the implementation of the password checker, and
as such, we are always sure that our probabilistic model corresponds to
our actual model of a given situation.

After the modeling, we can attempt to derive the entropy. To make it feasible
we decided to reduce to the domain of the strings to two characters. Already at three
characters, the queries took too long to be feasible on our computers with a
sequential implementation.

%Min Entropy
%0.9911242603550298
%Shannon Entropy
%9.618774814425457e-2

The derived Shannon entropy was roughly \textbf{0.09618} while the min-entropy
yielded approximately \textbf{0.99112}. This difference indicates, as expected that
the min-entropy reports a more conservative measure than the Shannon entropy.

\textbf{A remark on modeling and sampling:} When implementing this example I
had to implement uniform distributions over strings. Here we have a couple of
ways to do it. The naive way would be something like following.

\begin{figure}[H]
\begin{center}
\begin{minted}{haskell}
pThreeLetterStringOld = uniform [[x1, x2, x3] | 
    x1 <- ['a' .. 'z'],
    x2 <- ['a' .. 'z'],
    x3 <- ['a' .. 'z']]
\end{minted}
\end{center}
\caption{An inefficient implementation of a uniform distribution over a 3 character string}
\label{code:uniform-string-slow}
\end{figure}

The idea is to make a list of all three-letter strings and then make it to a
uniform distribution.
This construction is not efficient for sampling as we get a list of 17576
elements we potentially have to run through.

\begin{figure}[H]
\begin{center}
\begin{minted}{haskell}
pStringNormal = do
    x1 <- uniform ['a' .. 'z']
    x2 <- uniform ['a' .. 'z']
    x3 <- uniform ['a' .. 'z']
    return [x1, x2, x3]
\end{minted}
\end{center}
\caption{An efficient implementation of a uniform distribution over a 3 character string}
\label{code:uniform-string-fast}
\end{figure}

An alternative implementation in this setting would be the one in Figure \ref{code:uniform-string-fast}.
This method is strictly faster to sample from, as we have three lists of 26 elements
summing to a maximum of 78 elements.

The speed increase does have a cost as we use three times more randomness.

Another thing we also have to note is that exact expectation querying takes
the same time in both implementations. However, if we do keep the recommendations
about sampling in mind, the second implementation is preferred.

% The Foundational Cryptography Framework
\section{Cryptography}
We have already discussed propositional reasoning using probabilities.
These techniques are applied to numerous proofs within cryptography.
For a comprehensive walk through on the examples, we refer to \cite{Petcher:2015}.

\bibliographystyle{model1-num-names}
\bibliography{bibliography.bib}


\end{document}

%%
%% End of file `elsarticle-template-1-num.tex'.
